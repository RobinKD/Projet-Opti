\documentclass[sigconf]{acmart}

\usepackage{booktabs} % For formal tables
\usepackage{graphicx}
\usepackage{rotating}
\definecolor{Gray}{gray}{0.6}
\usepackage{tabularx}
\usepackage{xspace}
\usepackage{float}
\usepackage{xstring} % for string operations
\usepackage{wasysym} % Table legend with symbols input from post-processing
\usepackage{MnSymbol} % Table legend with symbols input from post-processing
\usepackage{ifthen}
\usepackage{diagbox}
\usepackage{url}

\settopmatter{printacmref=false}
\usepackage{amsmath}
\usepackage{algorithm}% http://ctan.org/pkg/algorithms
\usepackage{algpseudocode}% http://ctan.org/pkg/algorithmicx
\algrenewcommand\algorithmicrequire{\textbf{Input:}}
\algrenewcommand\algorithmicensure{\textbf{Output:}}
\algrenewcommand\algorithmicfunction{\textbf{Step}}


% Copyright
%\setcopyright{none}
%\setcopyright{acmcopyright}
%\setcopyright{acmlicensed}
%\setcopyright{rightsretained}
%\setcopyright{usgov}
%\setcopyright{usgovmixed}
%\setcopyright{cagov}
%\setcopyright{cagovmixed}

% DOI
\acmDOI{None}

% ISBN
\acmISBN{None}

%Conference
\acmConference[]{None}{}{}
\acmYear{}
\copyrightyear{}

\acmPrice{}



%%%%%%%%%%%%%%%%%%%%%%   END OF PREAMBLE   %%%%%%%%%%%%%%%%%%%%%%%%%%%%%%%%%%%%

%%%%%%%%%%%%%%%%%%%%%%%%%%%%%%%%%%%%%%%%%%%%%%%%%%%%%%%%%%%%%%%%%%%%%%%%%%%%%%%
%%%%%%%%% TO BE EDITED %%%%%%%%%%%%%%%%%%%%%%%%%%%%%%%%%%%%%%%%%%%%%%%%%%%%%%%%
%%%%%%%%%%%%%%%%%%%%%%%%%%%%%%%%%%%%%%%%%%%%%%%%%%%%%%%%%%%%%%%%%%%%%%%%%%%%%%%

% Algorithm names as they appear in the tables, uncomment and adapt if necessary
 \newcommand{\algAtables}{ALGO1}  % first argument in the post-processing
% \newcommand{\algBtables}{ALGO2}  % second argument in the post-processing
% \newcommand{\algCtables}{ALGO3}  % third argument in the post-processing
% \newcommand{\algDtables}{ALGO4}  % forth argument in the post-processing
% ...
% location of pictures files
\newcommand{\bbobdatapath}{ppdata/} % change default output folder of COCO if desired
\input{\bbobdatapath cocopp_commands.tex}
\graphicspath{{\bbobdatapath\algsfolder}}

%%%%%%%%%%%%%%%%%%%%%%%%%%%%%%%%%%%%%%%%%%%%%%%%%%%%%%%%%%%%%%%%%%%%%%%%%%%%%%%
%%%%%%%%%%%%%%%%%%%%%%%%%%%%%%%%%%%%%%%%%%%%%%%%%%%%%%%%%%%%%%%%%%%%%%%%%%%%%%%
%%%%%%%%%%%%%%%%%%%%%%%%%%%%%%%%%%%%%%%%%%%%%%%%%%%%%%%%%%%%%%%%%%%%%%%%%%%%%%%

\newcommand{\DIM}{\ensuremath{\mathrm{DIM}}}
\newcommand{\aRT}{\ensuremath{\mathrm{aRT}}}
\newcommand{\FEvals}{\ensuremath{\mathrm{FEvals}}}
\newcommand{\nruns}{\ensuremath{\mathrm{Nruns}}}
\newcommand{\Dfb}{\ensuremath{\Delta f_{\mathrm{best}}}}
\newcommand{\Df}{\ensuremath{\Delta f}}
\newcommand{\DI}{\ensuremath{\Delta I_{\mathrm{HV}}^{\mathrm{COCO}}}}
\newcommand{\nbFEs}{\ensuremath{\mathrm{\#FEs}}}
\newcommand{\ftarget}{\ensuremath{f_\mathrm{t}}}
\newcommand{\Itarget}{\ensuremath{I_\mathrm{target}}}
\newcommand{\CrE}{\ensuremath{\mathrm{CrE}}}
\newcommand{\hvref}{\ensuremath{I_\mathrm{ref}}}
\newcommand{\fopt}{\hvref}
\newcommand{\change}[1]{{\color{red} #1}}
\newcommand{\TODO}[1]{{\color{orange} !!! #1 !!!}}
\newcommand{\bbobbiobj}{{\ttfamily bbob-biobj}\xspace}


%%%%%%%%%%%%%%%%%%%%%%   END OF PREAMBLE   %%%%%%%%%%%%%%%%%%%%%%%%%%%%%%%%%%%%
\setcopyright{none}
\begin{document}

\title{Black-Box Optimization Benchmarking of the \\Multiobjective Optimizer adaptive IBEA ($\epsilon$ -indicator) \\with the COCO platform}
\renewcommand{\shorttitle}{Template to Compare Multiple Algorithms on the \bbobbiobj Testbed}
%\titlenote{Submission deadline: March 31st.}
%Camera-ready paper due April 24th.}}
%\subtitle{Draft version}



\author{Martin BAUW}
%\authornote{tba if needed}
%\orcid{1234-5678-9012}
%\affiliation{%
%  \institution{Institute for Clarity in Documentation}
%  \streetaddress{P.O. Box 1212}
%  \city{Dublin} 
%  \state{Ohio} 
%  \postcode{43017-6221}
%}
%\email{trovato@corporation.com}
%
\author{Robin DURAZ}
%\authornote{The secretary disavows any knowledge of this author's actions.}
%\affiliation{%
%  \institution{Institute for Clarity in Documentation}
%  \streetaddress{P.O. Box 1212}
%  \city{Dublin} 
%  \state{Ohio} 
%  \postcode{43017-6221}
%}
%\email{webmaster@marysville-ohio.com}
%
\author{Jiaxin GAO}
%\authornote{This author is the
%  one who did all the really hard work.}
%\affiliation{%
%  \institution{The Th{\o}rv{\"a}ld Group}
%  \streetaddress{1 Th{\o}rv{\"a}ld Circle}
%  \city{Hekla} 
%  \country{Iceland}}
%\email{larst@affiliation.org}
%
%\author{Lawrence P. Leipuner}
%\affiliation{
%  \institution{Brookhaven Laboratories}
%  \streetaddress{P.O. Box 5000}}
%\email{lleipuner@researchlabs.org}
%
%\author{Sean Fogarty}
%\affiliation{%
%  \institution{NASA Ames Research Center}
%  \city{Moffett Field}
%  \state{California} 
%  \postcode{94035}}
%\email{fogartys@amesres.org}
%
\author{Hao LIU}
%\affiliation{%
%  \institution{Palmer Research Laboratories}
%  \streetaddress{8600 Datapoint Drive}
%  \city{San Antonio}
%  \state{Texas} 
%  \postcode{78229}}
%\email{cpalmer@prl.com}
%
\author{Luca VEYRON-FORRER}
%\affiliation{\institution{The Th{\o}rv{\"a}ld Group}}
%\email{jsmith@affiliation.org}
%
%\author{Julius P.~Kumquat}
%\affiliation{\institution{The Kumquat Consortium}}
%\email{jpkumquat@consortium.net}

% The default list of authors is too long for headers}
\renewcommand{\shortauthors}{Firstname Lastname et. al.}


\begin{abstract}
This paper discusses the implementation of a multiobjective evolutionary algorithm (MOEA). We implemented the Multiobjective Optimizer IBEA (indicator-based evolutionary algorithm) with $\epsilon$-indicator \cite{IBEA_base} in Python and benchmarked it using the COCO platform\cite{hansen2016exp}. In addition, we compared the results obtained with this algorithm with those of NSGA 2 and Random Search. 
\end{abstract}


%
% The code below should be generated by the tool at
% http://dl.acm.org/ccs.cfm
% Please copy and paste the code instead of the example below. 
%
% \begin{CCSXML}
%<ccs2012>
%<concept>
%<concept_id>10010147.10010178.10010205.10010208</concept_id>
%<concept_desc>Computing methodologies~Continuous space search</concept_desc>
%<concept_significance>500</concept_significance>
%</concept>
%</ccs2012>
%\end{CCSXML}

%\ccsdesc[500]{Computing methodologies~Continuous space search}


% We no longer use \terms command
%\terms{Algorithms}

% Complete with anything that is needed
\keywords{Evolutionary algorithm, Benchmarking, Black-box optimization, Bi-objective optimization, Decision Vector}

\maketitle


\section{Introduction}

The adaptive IBEA we implement is dedicated to the following improvements: it requires no diversity preservation techniques within its population evolution mechanism, and it aims at taking into account arbitrary preference information thanks to its $\epsilon$ additive binary quality indicators. It operates on a set of candidates in a decision space, which will be modified thanks to selection and variation mechanisms. 

The parallel with evolution can be reminded here: selection in our algorithm could be associated with the competition for reproduction and resources in nature, while variation illustrates the ability of existing living creatures to create new living beings thanks to genetic recombination and mutation. Note that due to the randomness of the variation process some individuals may not be transformed in the evolutionary mechanism.

Since we are in the case of multiobjective optimization, it is possible that several optimal objective vectors co-exist: they would be trade-offs between the different objectives we are pursuing. As we will discover in the algorithm description, our IBEA could, as any general stochastic search algorithm, be divided into three main elements: a working memory (the population of decision space candidates), a selection module, and a variation module. Among the differences with single-objective optimization, the working memory can here consider several solutions at a time. In single-objective optimizatin no mating selection is required and variation translates into modifying the current solution candidate.

\subsection{Why binary quality indicators ?}

Unary quality measures have been proved to be theoretically limited: they do not allow to determine whether a Pareto set approximation is better than another one. This limitation is still valid for a finite combinatio of unary quality measures. Binary quality indicators overcome part of the limitations and can, for certain binary indicators, indicate whether a Pareto approximation set is better than another one. \cite{IBEA_tutorial}

\section{Description of algorithm}

\subsection{Description of the objects associated with the algorithm}

The algorithm input consists of decision space vectors $x^i \in X$, an objective space $Z$ and objective functions $f^j : X \rightarrow Z$. We suppose that $X \subseteq \mathbb{R}^l$ with $l \in \{2,3,5,10,20,40\}$ and $Z \subseteq \mathbb{R}^2$. 

The output of a MOEA is a set of incomparable decision vector, meaning that no member of the output set dominates another one. Domination between decision vectors is defined as follows: a decision vector $x^1$ is said to dominate another decision vector $x^2$ ($ x^1 \succ x^2$), if $f_i (x^1) \leq f_i(x^2) \forall i \in \{1,...,n\}$ and $\exists j \in \{1,...n\}$ for which $f_j(x^1) < f_j(x^2)$.

This output will be our Pareto set approximation, and the space of Pareto set approximations will be noted as $\Omega$. Our algorithm relies on a binary quality indicator $I: \Omega \times \Omega \rightarrow \mathbb{R}$ which associates a real number to $k$ Pareto set approximations.

Our binary quality indicator is defined as the minimum distance by which a Pareto set approximation needs to be translated in each dimension in objective space so that another approximation (image of a decision space vector) is weakly dominated. The weak domination is defined as follows: decision vector $x^1$ weakly dominates $x^2$, written $x^1 \succeq x^2$, if $x^1$ dominates $x^2$ or the corresponding objective vectors are equal \cite{IBEA_base}. The mathematical translation of this definition consists in the following equation:

\begin{equation}
I_{\epsilon^+ (A,B)} = min_{\epsilon} \{\forall x^2 \in B\ \exists x^1 \in A : f_i(x^1) - \epsilon \leq f_i (x^2) for\ i \in \{1,...,n\}\}
\end{equation}

\subsection{Description of the algorithm adaptive IBEA}

The algorithm steps is described in pseudo-code ~\ref{alg:ALG1}. The adaptive version of IBEA answers the potential issue of widely spread binary quality indicators values. Too spread out indicators values complicate the task of determining a correct value for $K$, the scaling factor associated with our indicator. By adaptivaly scaling the binary quality indicators values back to a common $[-1;1]$ interval, we substantially suppress the need to adapt $K$ to our different problems.

\begin{algorithm}%[H]
        \caption{Adaptive IBEA}
        \label{alg:ALG1}
        \begin{algorithmic}
        \Require  \\ 
        		$\alpha$  (population size)\\
        		$N$ (maximum number of generations)\\
        		$K$ (fitness scaling factor)\\
        \Ensure \\
        		$A$ (Pareto set approximation)
        \Function {1. Initialization}{}
%            \If {  $wrejkwe$ ($rw$) trwer tewwerl }
%            %       \COMMENT { 
%            \State {jklrjkljfgkljlkj  kjkldfj gfdsdf }
%            \State  Set fdgsdsd
%            \ForAll  {$j=1$ to $N (x)$}
%            \State        Call $fgsd(x)$  
%            \State        Set $sfgdfgd =sfdg + fgds $ 
%
%            \EndFor 
%            \EndIf
		\State generate initial population of size $\alpha$
		\State set generation counter $m$ to $0$
        \EndFunction
        \Function {2. Fitness assignment}{}
        	\ForAll {objective function $f_i$}
        	\State		lower bound \underline{$b_i$} = $min_{x \in P} f_i (x)$
        	\State
        	\State		upper bound $\overline{b_i} = max_{x \in P} f_i(x)$
        	\EndFor
			\ForAll {objective function $f_i$}
			\State		$f^{'}_i(x) = \dfrac{f_i(x) - \underline{b_i}}{\overline{b_i} - \underline{b_i}}$
			\EndFor
			\State calculate all indicator values $I(x^1,x^2)$ with $f^{'}_i$
			\State determine max. indicator $c=max_{x^1,x^2 \in P} |I(x^1,x^2)|$
        	\ForAll {$x^1 \in P$}
        	\State 		$F(x^1) = \sum_{x^2 \in P \backslash \{x^1\}} -e^{-\dfrac{I(\{x^1\},\{x^2\})}{ck}}$
        	\EndFor
        \EndFunction
        \Function {3. Environmental selection}{}
        	\While {population P $\geq \alpha$}
        	\State 		choose $x^{*}$ such that $F(x^{*}) \leq F(x)$ for all $x \in P$.
        	\State		remove $x^{*}$ from the population
        	\State		update remaining individuals fitness values, \textit{ie} $\forall x \in P$:
        	\State 		$F(x) = F(x) + e^{-\dfrac{I(\{x^{*}\},\{x\})}{ck}}$
        	\EndWhile
        \EndFunction
        \Function {4. Termination}{}
        	\State If $m \geq N$ or another stopping criterion then the output $A$ is
			\State defined as the set of nondominated decision vectors in $P$
        \EndFunction
        \Function {5. Mating selection}{}
        	\State Binary tournament selection with replacement on $P$ in
        	\State order to fill a temporary mating pool $P'$
        \EndFunction
        \Function {6. Variation}{}
        	\State Apply recombination and mutation operators to $P'$
        	\State Add the resulting offspring to P
        	\State Increment the counter ($m=m+1$) and go to Step 2
        \EndFunction
        \end{algorithmic}{}
    \end{algorithm}

IBEA can be faster than other algorithms since it only compares pair of decision vectors rather than complete approximation sets.

\subsection{Details regarding mating selection}

Mating selection aims at picking promising solutions for variation and usually is performed in a randomized fashion. Perform binary tournament selection with replacement on $P$ in order to fill a temporary mating pool (the variable $P\_$ in our implementation). The binary tournament consists in keeping the candidate with the best fitness value from the random pick.

\subsection{Details regarding mutation}

The mutation operator modifies individuals by changing small parts in the associated vectors according to a given mutation rate. The mutation rate determines how much of the considered population we will use to generate new decision space vectors. This process reminds how related evolutionary algorithms seem related to biological evolution, mimicking natural hereditary relationships. For mutation we use a polynomial mutation operator.

The operations involved in one single mutation are as follows. A random individual is picked from the population $P$. According to a uniform probability pick, the mutation is realized using one mathematical transformation applied to its coordinates in the decision space. Note that the transformation change from one coordinate to another, since the uniform probability pick is repeated for each. In our case, there are two forms of transformations on coordinates. In the following equations, $u$ is uniformly picked in $[0,1]$, $ind[j]$ the $j-th$ coordinate of the already existing picked individual, $p_{mut}[j]$ the $j-th$ coordinate of the newly generated individual, $Up$ the biggest existing coordinate value, $Lo$ the lowest. The two latters are each defined for each decision space dimension.

\begin{itemize}
\item if the uniform probability pick $\leq 0.5$:
\begin{equation}
\sigma_L = (2u)^{\frac{1}{\mu +1}}-1 
\end{equation}
\begin{equation}
p_{mut}[j] = ind[j] + \sigma_L(ind[j]-Lo)
\end{equation}
\item else:
\begin{equation}
\sigma_R = (2(1-u))^{\frac{1}{\mu +1}} 
\end{equation}
\begin{equation}
p_{mut}[j] = ind[j] + \sigma_R(Up-ind[j])
\end{equation}
\end{itemize}

This is implemented thanks to a loop on individuals, with a nested loop treating each decision space dimension coordinate. Mathematical formulas involving hyperparameters are based on \cite{compute_sigma}.

% https://www.iitk.ac.in/kangal/papers/k2012016.pdf

\subsection{Details regarding recombination}

The recombination operator takes a certain number of parents and creates a predefined number of children by combining parts of the parents. For recombination we use a simulated binary crossover (SBX) operator. To mimic the stochastic nature of evolution, a crossover probability is associated with this operator. In our case, two parents are selected among the current population. A uniform probability pick in $[0,1]$ written $u$ determines the parameter used in computing the features (decision space coordinates) of the children. In the following equations, $child0-1[j]$ is the $j-th$ coordinate of the generated child decision vector, $parent0-1[j]$ the $j-th$ coordinate of the parent associated with the child.

\begin{itemize}
\item if the uniform probability pick $\leq 0.5$:
\begin{equation}
\beta_q = (2u)^{\frac{1}{\mu +1}}
\end{equation}
\item else:
\begin{equation}
\beta_q = (\frac{1}{2(1-u)})^{\frac{1}{\mu +1}}
\end{equation}
\end{itemize}

Thanks to this stochastic parameter we can compute the children's coordinates:

\begin{itemize}
\item first child:
\begin{equation}
child0[j] = 0.5((1+\beta_q)parent0[j]+(1-\beta_q)parent1[j])
\end{equation}
\item second child:
\begin{equation}
child1[j] = 0.5((1-\beta_q)parent0[j]+(1+\beta_q)parent1[j])
\end{equation}
\end{itemize}

Regarding the hyperparameter $\mu$, formulas were taken from \cite{compute_mu}. In the article this parameter is referred to as $\eta_c$. A large $\eta_c$ implies an offspring close to the parents in coordinates. For a smaller $\eta_c$, children solutions tend to differ more from their parents. This parameter is therefore essential to controlling the spread of the offspring.
% http://www.cs.bham.ac.uk/~wbl/biblio/gecco2007/docs/p1187.pdf

\section{Description of implementation}

%%%%%%%%%%%%%%%%%%%%%%%%%%%%%%%%%%%%%%%%%%%%%%%%%%%%%%%%%%%%%%%%%%%%%%%%%%%%%%%
\section{CPU Timing}
%%%%%%%%%%%%%%%%%%%%%%%%%%%%%%%%%%%%%%%%%%%%%%%%%%%%%%%%%%%%%%%%%%%%%%%%%%%%%%%
%note that the following text is just a proposal and can/should be changed to your needs:
In order to evaluate the CPU timing of the algorithm, we have run the %\change{\algorithmA} 
with restarts on the entire bbob-biobj test suite \cite{biobj2016func} for $2 D$ function evaluations according to \cite{hansen2016exp}. The Python code was run on a \change{Mac Intel(R) Core(TM) i5-2400S CPU @ 2.50GHz} with \change{1} processor and \change{4} cores \change{and (compile) options xxx}. The time per function evaluation for dimensions 2, 3, 5, 10, 20\change{, 40} equals \change{$x.x$}, \change{$x.x$}, \change{$x.x$}, \change{$xx$}, \change{$xxx$}\change{, and $xxx$} seconds respectively. 

\change{repeat the above for any algorithm tested}

%%%%%%%%%%%%%%%%%%%%%%%%%%%%%%%%%%%%%%%%%%%%%%%%%%%%%%%%%%%%%%%%%%%%%%%%%%%%%%%
\section{Discussion of the results}
%%%%%%%%%%%%%%%%%%%%%%%%%%%%%%%%%%%%%%%%%%%%%%%%%%%%%%%%%%%%%%%%%%%%%%%%%%%%%%%

Results from experiments according to \cite{hansen2016exp},
\cite{hansen2016perfass} and \cite{biobj2016perfass} on the benchmark
functions given in \cite{biobj2016func} are presented in
Figures~\ref{fig:ECDFsingleOne}, \ref{fig:ECDFsingleTwo}, \ref{fig:ECDFsGroupsFive} and
\ref{fig:ECDFsGroupsTwenty} and in Tables~\ref{tab:aRTs5} and~\ref{tab:aRTs20}.
The experiments were performed with COCO \cite{hansen2016cocoplat}, version
\change{2.0}, the plots were produced with version \change{2.0}.

The \textbf{average runtime (aRT)}, used in the %figures and
tables,
depends on a given quality indicator value, $\Itarget=\hvref+\DI$, and is
computed over all relevant trials as the number of function
evaluations executed during each trial while the best indicator value
did not reach \Itarget, summed over all trials and divided by the
number of trials that actually reached \Itarget\
\cite{hansen2016exp,price1997dev}.  \textbf{Statistical significance}
is tested with the rank-sum test for a given target $\Itarget$
using, for each trial,
either the number of needed function evaluations to reach
$\Itarget$ (inverted and multiplied by $-1$), or, if the target
was not reached, the best $\DI$-value achieved, measured only up to
the smallest number of overall function evaluations for any
unsuccessful trial under consideration.

\subsection{Results of our algorithm}

\subsection{Comparison with NSGA 2 and Random Search}
NSGA-II is a very famous multi-objective optimization algorithm. It has three special characteristics: fast non-dominated sorting approach, fast crowded distance estimation procedure and simple crowded comparison operator. And it has been used successfully in lots of multi-objective optimization problems. I take the data obtained by NSGA-II algorithm from the following link: \\ \url{http://coco.gforge.inria.fr/data-archive/bbob-biobj/2016/NSGA-II-MATLAB_Auger_bbob-biobj.tgz} \\
\\
Here is the result of the NSGA-II algorithm: \\
% ici image pprldmany de NSGA-II
The basic idea of the Random search algorithm is to randomly select some points according to certain criteria, then put these points into the functions, and keep the best points from them for the next iteration. It is essentially a violent search mechanism, but when the number of evalutions is greatly increased, it can also achieve good results. I take the data obtained by Random search algorithm from the following link: \\
\url{http://coco.gforge.inria.fr/data-archive/bbob-biobj/2016/RANDOMSEARCH-5_Auger_bbob-biobj.tgz} \\
Here is the result of the Random search algorithm: \\
% ici image pprldmany de Randomsearch
\begin{table}[!htbp]
%\centering
\caption{Comparison of three algorithms}

\begin{tabular}{|p{3cm}|p{2.6cm}|p{2cm}|}
\hline
\diagbox{Algorithm}{Category}&Log10(f-evals/ dimension)&Best for 2-D\\
\hline
Random search&6&0,62\\
\hline
NSGA-II&5&0,76\\
\hline
IBEA& 2,4(40D)-3,6(2D)&0,58\\
\hline
\end{tabular}
\begin{tabular}{|p{2.4cm}|p{2.6cm}|p{2.6cm}|}
\hline
Best for 40-D&Best evaluated function&Worst evaluated function\\
\hline
0,08&f13, f53&f37, f46\\
\hline
0,17&f19, f53&f22\\
\hline
0,12&f18, f53&f31, f47\\
\hline
\end{tabular}
\end{table}


From the above table we can see that, in general, NSGA-II is the best algorithm, its evaluation is not too big, but it can get the best results. IBEA is also a good algorithm, it does not need too many evaluations, but it can also get good results. And according to the estimation of COCO, its best value can exceed Random search (if the algorithm does more evaluations). In contrast, the effect of Random search is not very good, it need too many evaluations to get a good value, and when the dimension of input becomes higher, the algorithm is not so effective. From the last two columns of the table, we can see that different algorithms have different optimization capabilities for each function, and f53 is friendly to all the algorithms.

%Maybe add the others images of NSGA-II and Randome search in the appendix
\section{Conclusion}

%%%%%%%%%%%%%%%%%%%%%%%%%%%%%%%%%%%%%%%%%%%%%%%%%%%%%%%%%%%%%%%%%%%%%%%%%%%%%%%
%%%%%%%%%%%%%%%%%%%%%%%%%%%%%%%%%%%%%%%%%%%%%%%%%%%%%%%%%%%%%%%%%%%%%%%%%%%%%%%

% ECDFs per function in dimension 10

%%%%%%%%%%%%%%%%%%%%%%%%%%%%%%%%%%%%%%%%%%%%%%%%%%%%%%%%%%%%%%%%%%%%%%%%%%%%%%%
%\begin{figure*}
%\centering
%\begin{tabular}{@{\hspace*{-0.00\textwidth}}l@{\hspace*{-0.00\textwidth}}l@{\hspace*{-0.00\textwidth}}l@{\hspace*{-0.00\textwidth}}l@{\hspace*{-0.00\textwidth}}l@{\hspace*{-0.00\textwidth}}}
%\includegraphics[width=0.2\textwidth]{pprldmany-single-functions/pprldmany_f001_10D}&
%\includegraphics[width=0.2\textwidth]{pprldmany-single-functions/pprldmany_f002_10D}&
%\includegraphics[width=0.2\textwidth]{pprldmany-single-functions/pprldmany_f003_10D}&
%\includegraphics[width=0.2\textwidth]{pprldmany-single-functions/pprldmany_f004_10D}&
%\includegraphics[width=0.2\textwidth]{pprldmany-single-functions/pprldmany_f005_10D}\\
%\includegraphics[width=0.2\textwidth]{pprldmany-single-functions/pprldmany_f006_10D}&
%\includegraphics[width=0.2\textwidth]{pprldmany-single-functions/pprldmany_f007_10D}&
%\includegraphics[width=0.2\textwidth]{pprldmany-single-functions/pprldmany_f008_10D}&
%\includegraphics[width=0.2\textwidth]{pprldmany-single-functions/pprldmany_f009_10D}&
%\includegraphics[width=0.2\textwidth]{pprldmany-single-functions/pprldmany_f010_10D}\\
%\includegraphics[width=0.2\textwidth]{pprldmany-single-functions/pprldmany_f011_10D}&
%\includegraphics[width=0.2\textwidth]{pprldmany-single-functions/pprldmany_f012_10D}&
%\includegraphics[width=0.2\textwidth]{pprldmany-single-functions/pprldmany_f013_10D}&
%\includegraphics[width=0.2\textwidth]{pprldmany-single-functions/pprldmany_f014_10D}&
%\includegraphics[width=0.2\textwidth]{pprldmany-single-functions/pprldmany_f015_10D}\\
%\includegraphics[width=0.2\textwidth]{pprldmany-single-functions/pprldmany_f016_10D}&
%\includegraphics[width=0.2\textwidth]{pprldmany-single-functions/pprldmany_f017_10D}&
%\includegraphics[width=0.2\textwidth]{pprldmany-single-functions/pprldmany_f018_10D}&
%\includegraphics[width=0.2\textwidth]{pprldmany-single-functions/pprldmany_f019_10D}&
%\includegraphics[width=0.2\textwidth]{pprldmany-single-functions/pprldmany_f020_10D}\\
%\includegraphics[width=0.2\textwidth]{pprldmany-single-functions/pprldmany_f021_10D}&
%\includegraphics[width=0.2\textwidth]{pprldmany-single-functions/pprldmany_f022_10D}&
%\includegraphics[width=0.2\textwidth]{pprldmany-single-functions/pprldmany_f023_10D}&
%\includegraphics[width=0.2\textwidth]{pprldmany-single-functions/pprldmany_f024_10D}&
%\includegraphics[width=0.2\textwidth]{pprldmany-single-functions/pprldmany_f025_10D}\\
%\includegraphics[width=0.2\textwidth]{pprldmany-single-functions/pprldmany_f026_10D}&
%\includegraphics[width=0.2\textwidth]{pprldmany-single-functions/pprldmany_f027_10D}&
%\includegraphics[width=0.2\textwidth]{pprldmany-single-functions/pprldmany_f028_10D}&
%\includegraphics[width=0.2\textwidth]{pprldmany-single-functions/pprldmany_f029_10D}&
%\includegraphics[width=0.2\textwidth]{pprldmany-single-functions/pprldmany_f030_10D}\\
%\includegraphics[width=0.2\textwidth]{pprldmany-single-functions/pprldmany_f031_10D}&
%\includegraphics[width=0.2\textwidth]{pprldmany-single-functions/pprldmany_f032_10D}&
%\includegraphics[width=0.2\textwidth]{pprldmany-single-functions/pprldmany_f033_10D}&
%\includegraphics[width=0.2\textwidth]{pprldmany-single-functions/pprldmany_f034_10D}&
%\includegraphics[width=0.2\textwidth]{pprldmany-single-functions/pprldmany_f035_10D}\\[-1.8ex]
%\end{tabular}
% \caption{\label{fig:ECDFsingleOne}
%	\bbobecdfcaptionsinglefunctionssingledim{10}
%}
%\end{figure*}
%
%\begin{figure*}
%\centering
%\begin{tabular}{@{\hspace*{-0.00\textwidth}}l@{\hspace*{-0.00\textwidth}}l@{\hspace*{-0.00\textwidth}}l@{\hspace*{-0.00\textwidth}}l@{\hspace*{-0.00\textwidth}}l@{\hspace*{-0.00\textwidth}}}
%\includegraphics[width=0.2\textwidth]{pprldmany-single-functions/pprldmany_f036_10D}&
%\includegraphics[width=0.2\textwidth]{pprldmany-single-functions/pprldmany_f037_10D}&
%\includegraphics[width=0.2\textwidth]{pprldmany-single-functions/pprldmany_f038_10D}&
%\includegraphics[width=0.2\textwidth]{pprldmany-single-functions/pprldmany_f039_10D}&
%\includegraphics[width=0.2\textwidth]{pprldmany-single-functions/pprldmany_f040_10D}\\
%\includegraphics[width=0.2\textwidth]{pprldmany-single-functions/pprldmany_f041_10D}&
%\includegraphics[width=0.2\textwidth]{pprldmany-single-functions/pprldmany_f042_10D}&
%\includegraphics[width=0.2\textwidth]{pprldmany-single-functions/pprldmany_f043_10D}&
%\includegraphics[width=0.2\textwidth]{pprldmany-single-functions/pprldmany_f044_10D}&
%\includegraphics[width=0.2\textwidth]{pprldmany-single-functions/pprldmany_f045_10D}\\
%\includegraphics[width=0.2\textwidth]{pprldmany-single-functions/pprldmany_f046_10D}&
%\includegraphics[width=0.2\textwidth]{pprldmany-single-functions/pprldmany_f047_10D}&
%\includegraphics[width=0.2\textwidth]{pprldmany-single-functions/pprldmany_f048_10D}&
%\includegraphics[width=0.2\textwidth]{pprldmany-single-functions/pprldmany_f049_10D}&
%\includegraphics[width=0.2\textwidth]{pprldmany-single-functions/pprldmany_f040_10D}\\
%\includegraphics[width=0.2\textwidth]{pprldmany-single-functions/pprldmany_f051_10D}&
%\includegraphics[width=0.2\textwidth]{pprldmany-single-functions/pprldmany_f052_10D}&
%\includegraphics[width=0.2\textwidth]{pprldmany-single-functions/pprldmany_f053_10D}&
%\includegraphics[width=0.2\textwidth]{pprldmany-single-functions/pprldmany_f054_10D}&
%\includegraphics[width=0.2\textwidth]{pprldmany-single-functions/pprldmany_f055_10D}\\[-1.8ex]
%\end{tabular}
% \caption{\label{fig:ECDFsingleTwo}
% Bootstrapped empirical cumulative distribution of the number of objective function evaluations divided by dimension (FEvals/DIM) as in Fig.~\ref{fig:ECDFsingleOne} but for functions $f_{36}$ to $f_{55}$ in 10-D.
%}
%\end{figure*}
%
%
%
%
%%%%%%%%%%%%%%%%%%%%%%%%%%%%%%%%%%%%%%%%%%%%%%%%%%%%%%%%%%%%%%%%%%%%%%%%%%%%%%%%
%%%%%%%%%%%%%%%%%%%%%%%%%%%%%%%%%%%%%%%%%%%%%%%%%%%%%%%%%%%%%%%%%%%%%%%%%%%%%%%%
%
%% Empirical cumulative distribution functions (ECDFs) per function group (5-D)
%
%%%%%%%%%%%%%%%%%%%%%%%%%%%%%%%%%%%%%%%%%%%%%%%%%%%%%%%%%%%%%%%%%%%%%%%%%%%%%%%%
%
%\begin{figure*}
%\begin{tabular}{@{\hspace*{-0.00\textwidth}}c@{\hspace*{-0.0\textwidth}}c@{\hspace*{-0.0\textwidth}}c@{\hspace*{-0.0\textwidth}}c}
%separable-separable & separable-moderate & separable-ill-cond. & separable-multimodal\\
%\includegraphics[width=0.24\textwidth]{pprldmany_05D_1-separable_1-separable} &
%\includegraphics[width=0.24\textwidth]{pprldmany_05D_1-separable_2-moderate} &
%\includegraphics[width=0.24\textwidth]{pprldmany_05D_1-separable_3-ill-conditioned} &
%\includegraphics[width=0.24\textwidth]{pprldmany_05D_1-separable_4-multi-modal}\\
%separable-weakstructure & moderate-moderate & moderate-ill-cond. & moderate-multimodal\\
%\includegraphics[width=0.24\textwidth]{pprldmany_05D_1-separable_5-weakly-structured} &
%\includegraphics[width=0.24\textwidth]{pprldmany_05D_2-moderate_2-moderate} &
%\includegraphics[width=0.24\textwidth]{pprldmany_05D_2-moderate_3-ill-conditioned} &
%\includegraphics[width=0.24\textwidth]{pprldmany_05D_2-moderate_4-multi-modal}\\
%moderate-weakstructure & ill-cond.-ill-cond. & ill-cond.-multimodal & ill-cond.-weakstructure\\
%\includegraphics[width=0.24\textwidth]{pprldmany_05D_2-moderate_5-weakly-structured} &
%\includegraphics[width=0.24\textwidth]{pprldmany_05D_3-ill-conditioned_3-ill-conditioned} &
%\includegraphics[width=0.24\textwidth]{pprldmany_05D_3-ill-conditioned_4-multi-modal} &
%\includegraphics[width=0.24\textwidth]{pprldmany_05D_3-ill-conditioned_5-weakly-structured} \\
%multimodal-multimodal & multimodal-weakstructure & weakstructure-weakstructure & all 55 functions\\
%\includegraphics[width=0.24\textwidth]{pprldmany_05D_4-multi-modal_4-multi-modal} &
%\includegraphics[width=0.24\textwidth]{pprldmany_05D_4-multi-modal_5-weakly-structured} &
%\includegraphics[width=0.24\textwidth]{pprldmany_05D_5-weakly-structured_5-weakly-structured} &
%\includegraphics[width=0.24\textwidth]{pprldmany_05D_noiselessall}
%\vspace*{-0.5ex}
%\end{tabular}
% \caption{\label{fig:ECDFsGroupsFive}
% \bbobECDFslegend{5}
% }
%\end{figure*}
%
%%%%%%%%%%%%%%%%%%%%%%%%%%%%%%%%%%%%%%%%%%%%%%%%%%%%%%%%%%%%%%%%%%%%%%%%%%%%%%%%
%%%%%%%%%%%%%%%%%%%%%%%%%%%%%%%%%%%%%%%%%%%%%%%%%%%%%%%%%%%%%%%%%%%%%%%%%%%%%%%%
%
%% Empirical cumulative distribution functions (ECDFs) per function group (20-D)
%
%%%%%%%%%%%%%%%%%%%%%%%%%%%%%%%%%%%%%%%%%%%%%%%%%%%%%%%%%%%%%%%%%%%%%%%%%%%%%%%%
%
%\begin{figure*}
%\begin{tabular}{@{\hspace*{-0.00\textwidth}}c@{\hspace*{-0.0\textwidth}}c@{\hspace*{-0.0\textwidth}}c@{\hspace*{-0.0\textwidth}}c}
%separable-separable & separable-moderate & separable-ill-cond. & separable-multimodal\\
%\includegraphics[width=0.24\textwidth]{pprldmany_20D_1-separable_1-separable} &
%\includegraphics[width=0.24\textwidth]{pprldmany_20D_1-separable_2-moderate} &
%\includegraphics[width=0.24\textwidth]{pprldmany_20D_1-separable_3-ill-conditioned} &
%\includegraphics[width=0.24\textwidth]{pprldmany_20D_1-separable_4-multi-modal}\\
%separable-weakstructure & moderate-moderate & moderate-ill-cond. & moderate-multimodal\\
%\includegraphics[width=0.24\textwidth]{pprldmany_20D_1-separable_5-weakly-structured} &
%\includegraphics[width=0.24\textwidth]{pprldmany_20D_2-moderate_2-moderate} &
%\includegraphics[width=0.24\textwidth]{pprldmany_20D_2-moderate_3-ill-conditioned} &
%\includegraphics[width=0.24\textwidth]{pprldmany_20D_2-moderate_4-multi-modal}\\
%moderate-weakstructure & ill-cond.-ill-cond. & ill-cond.-multimodal & ill-cond.-weakstructure\\
%\includegraphics[width=0.24\textwidth]{pprldmany_20D_2-moderate_5-weakly-structured} &
%\includegraphics[width=0.24\textwidth]{pprldmany_20D_3-ill-conditioned_3-ill-conditioned} &
%\includegraphics[width=0.24\textwidth]{pprldmany_20D_3-ill-conditioned_4-multi-modal} &
%\includegraphics[width=0.24\textwidth]{pprldmany_20D_3-ill-conditioned_5-weakly-structured} \\
%multimodal-multimodal & multimodal-weakstructure & weakstructure-weakstructure & all 55 functions\\
%\includegraphics[width=0.24\textwidth]{pprldmany_20D_4-multi-modal_4-multi-modal} &
%\includegraphics[width=0.24\textwidth]{pprldmany_20D_4-multi-modal_5-weakly-structured} &
%\includegraphics[width=0.24\textwidth]{pprldmany_20D_5-weakly-structured_5-weakly-structured} &
%\includegraphics[width=0.24\textwidth]{pprldmany_20D_noiselessall}
%\vspace*{-0.5ex}
%\end{tabular}
% \caption{\label{fig:ECDFsGroupsTwenty}
% \bbobECDFslegend{20}
% }
%\end{figure*}
%
%
%\clearpage
%
%%%%%%%%%%%%%%%%%%%%%%%%%%%%%%%%%%%%%%%%%%%%%%%%%%%%%%%%%%%%%%%%%%%%%%%%%%%%%%%%
%%%%%%%%%%%%%%%%%%%%%%%%%%%%%%%%%%%%%%%%%%%%%%%%%%%%%%%%%%%%%%%%%%%%%%%%%%%%%%%%
%
%% Average runtime (aRT in number of function evaluations)
%% for functions $f_1$--$f_{55}$ of the bbob-biobj suite for dimension 5.
%
%%%%%%%%%%%%%%%%%%%%%%%%%%%%%%%%%%%%%%%%%%%%%%%%%%%%%%%%%%%%%%%%%%%%%%%%%%%%%%%%
%{\normalsize \color{red}
%\ifthenelse{\isundefined{\algorithmD}}{}{more than 3 algorithms: please split the tables below by hand until everything fits to the page limits}
%}
%
%\begin{table*}\tiny
%\centering
%\mbox{\begin{minipage}[t]{0.32\textwidth}\tiny
%\centering
%
%\pptablesheader
%
%\input{\bbobdatapath\algsfolder pptables_f001_05D} 
%
%\input{\bbobdatapath\algsfolder pptables_f002_05D}
%
%\input{\bbobdatapath\algsfolder pptables_f003_05D}
%
%\input{\bbobdatapath\algsfolder pptables_f004_05D}
%
%\input{\bbobdatapath\algsfolder pptables_f005_05D}
%
%\input{\bbobdatapath\algsfolder pptables_f006_05D}
%
%\input{\bbobdatapath\algsfolder pptables_f007_05D}
%
%\input{\bbobdatapath\algsfolder pptables_f008_05D}
%
%\input{\bbobdatapath\algsfolder pptables_f009_05D}
%
%\input{\bbobdatapath\algsfolder pptables_f010_05D}
%
%\input{\bbobdatapath\algsfolder pptables_f011_05D}
%
%\input{\bbobdatapath\algsfolder pptables_f012_05D}
%
%\input{\bbobdatapath\algsfolder pptables_f013_05D}
%
%\input{\bbobdatapath\algsfolder pptables_f014_05D}
%
%\input{\bbobdatapath\algsfolder pptables_f015_05D}
%
%\input{\bbobdatapath\algsfolder pptables_f016_05D}
%
%\input{\bbobdatapath\algsfolder pptables_f017_05D}
%
%\input{\bbobdatapath\algsfolder pptables_f018_05D}
%
%\input{\bbobdatapath\algsfolder pptables_f019_05D}
%
%\end{tabularx}
%
%\end{minipage}
%\hspace{0.002\textwidth}
%\begin{minipage}[t]{0.32\textwidth}\tiny
%\centering
%
%\pptablesheader
%
%\input{\bbobdatapath\algsfolder pptables_f020_05D}
%
%\input{\bbobdatapath\algsfolder pptables_f021_05D}
%
%\input{\bbobdatapath\algsfolder pptables_f022_05D}
%
%\input{\bbobdatapath\algsfolder pptables_f023_05D}
%
%\input{\bbobdatapath\algsfolder pptables_f024_05D}
%
%\input{\bbobdatapath\algsfolder pptables_f025_05D}
%
%\input{\bbobdatapath\algsfolder pptables_f026_05D}
%
%\input{\bbobdatapath\algsfolder pptables_f027_05D}
%
%\input{\bbobdatapath\algsfolder pptables_f028_05D}
%
%\input{\bbobdatapath\algsfolder pptables_f029_05D}
%
%\input{\bbobdatapath\algsfolder pptables_f030_05D}
%
%\input{\bbobdatapath\algsfolder pptables_f031_05D}
%
%\input{\bbobdatapath\algsfolder pptables_f032_05D}
%
%\input{\bbobdatapath\algsfolder pptables_f033_05D}
%
%\input{\bbobdatapath\algsfolder pptables_f034_05D}
%
%\input{\bbobdatapath\algsfolder pptables_f035_05D}
%
%\input{\bbobdatapath\algsfolder pptables_f036_05D}
%
%\input{\bbobdatapath\algsfolder pptables_f037_05D}
%
%\end{tabularx}
%
%\end{minipage}
%
%\hspace{0.002\textwidth}
%\begin{minipage}[t]{0.32\textwidth}\tiny
%\centering
%
%\pptablesheader
%
%\input{\bbobdatapath\algsfolder pptables_f038_05D}
%
%\input{\bbobdatapath\algsfolder pptables_f039_05D}
%
%\input{\bbobdatapath\algsfolder pptables_f040_05D}
%
%\input{\bbobdatapath\algsfolder pptables_f041_05D}
%
%\input{\bbobdatapath\algsfolder pptables_f042_05D}
%
%\input{\bbobdatapath\algsfolder pptables_f043_05D}
%
%\input{\bbobdatapath\algsfolder pptables_f044_05D}
%
%\input{\bbobdatapath\algsfolder pptables_f045_05D}
%
%\input{\bbobdatapath\algsfolder pptables_f046_05D}
%
%\input{\bbobdatapath\algsfolder pptables_f047_05D}
%
%\input{\bbobdatapath\algsfolder pptables_f048_05D}
%
%\input{\bbobdatapath\algsfolder pptables_f049_05D}
%
%\input{\bbobdatapath\algsfolder pptables_f050_05D}
%
%\input{\bbobdatapath\algsfolder pptables_f051_05D}
%
%\input{\bbobdatapath\algsfolder pptables_f052_05D}
%
%\input{\bbobdatapath\algsfolder pptables_f053_05D}
%
%\input{\bbobdatapath\algsfolder pptables_f054_05D}
%
%\input{\bbobdatapath\algsfolder pptables_f055_05D}
%
%\end{tabularx}
%
%\end{minipage}}
%
% \caption{\label{tab:aRTs5}
% \bbobpptablesmanylegend{dimension $5$}
% }
%\end{table*}
%%sideways
%
%
%%%%%%%%%%%%%%%%%%%%%%%%%%%%%%%%%%%%%%%%%%%%%%%%%%%%%%%%%%%%%%%%%%%%%%%%%%%%%%%%
%%%%%%%%%%%%%%%%%%%%%%%%%%%%%%%%%%%%%%%%%%%%%%%%%%%%%%%%%%%%%%%%%%%%%%%%%%%%%%%%
%
%% Average runtime (aRT in number of function evaluations)
%% for functions $f_1$--$f_{55}$ of the bbob-biobj suite for dimension 20.
%
%%%%%%%%%%%%%%%%%%%%%%%%%%%%%%%%%%%%%%%%%%%%%%%%%%%%%%%%%%%%%%%%%%%%%%%%%%%%%%%%
%\begin{table*}\tiny
%\centering
%\mbox{\begin{minipage}[t]{0.32\textwidth}\tiny
%\centering
%
%\pptablesheader
%
%\input{\bbobdatapath\algsfolder pptables_f001_20D} 
%
%\input{\bbobdatapath\algsfolder pptables_f002_20D}
%
%\input{\bbobdatapath\algsfolder pptables_f003_20D}
%
%\input{\bbobdatapath\algsfolder pptables_f004_20D}
%
%\input{\bbobdatapath\algsfolder pptables_f005_20D}
%
%\input{\bbobdatapath\algsfolder pptables_f006_20D}
%
%\input{\bbobdatapath\algsfolder pptables_f007_20D}
%
%\input{\bbobdatapath\algsfolder pptables_f008_20D}
%
%\input{\bbobdatapath\algsfolder pptables_f009_20D}
%
%\input{\bbobdatapath\algsfolder pptables_f010_20D}
%
%\input{\bbobdatapath\algsfolder pptables_f011_20D}
%
%\input{\bbobdatapath\algsfolder pptables_f012_20D}
%
%\input{\bbobdatapath\algsfolder pptables_f013_20D}
%
%\input{\bbobdatapath\algsfolder pptables_f014_20D}
%
%\input{\bbobdatapath\algsfolder pptables_f015_20D}
%
%\input{\bbobdatapath\algsfolder pptables_f016_20D}
%
%\input{\bbobdatapath\algsfolder pptables_f017_20D}
%
%\input{\bbobdatapath\algsfolder pptables_f018_20D}
%
%\input{\bbobdatapath\algsfolder pptables_f019_20D}
%
%\end{tabularx}
%
%\end{minipage}
%\hspace{0.002\textwidth}
%\begin{minipage}[t]{0.32\textwidth}\tiny
%\centering
%
%\pptablesheader
%
%\input{\bbobdatapath\algsfolder pptables_f020_20D}
%
%\input{\bbobdatapath\algsfolder pptables_f021_20D}
%
%\input{\bbobdatapath\algsfolder pptables_f022_20D}
%
%\input{\bbobdatapath\algsfolder pptables_f023_20D}
%
%\input{\bbobdatapath\algsfolder pptables_f024_20D}
%
%\input{\bbobdatapath\algsfolder pptables_f025_20D}
%
%\input{\bbobdatapath\algsfolder pptables_f026_20D}
%
%\input{\bbobdatapath\algsfolder pptables_f027_20D}
%
%\input{\bbobdatapath\algsfolder pptables_f028_20D}
%
%\input{\bbobdatapath\algsfolder pptables_f029_20D}
%
%\input{\bbobdatapath\algsfolder pptables_f030_20D}
%
%\input{\bbobdatapath\algsfolder pptables_f031_20D}
%
%\input{\bbobdatapath\algsfolder pptables_f032_20D}
%
%\input{\bbobdatapath\algsfolder pptables_f033_20D}
%
%\input{\bbobdatapath\algsfolder pptables_f034_20D}
%
%\input{\bbobdatapath\algsfolder pptables_f035_20D}
%
%\input{\bbobdatapath\algsfolder pptables_f036_20D}
%
%\input{\bbobdatapath\algsfolder pptables_f037_20D}
%
%\end{tabularx}
%
%\end{minipage}
%
%\hspace{0.002\textwidth}
%\begin{minipage}[t]{0.32\textwidth}\tiny
%\centering
%
%\pptablesheader
%
%\input{\bbobdatapath\algsfolder pptables_f038_20D}
%
%\input{\bbobdatapath\algsfolder pptables_f039_20D}
%
%\input{\bbobdatapath\algsfolder pptables_f040_20D}
%
%\input{\bbobdatapath\algsfolder pptables_f041_20D}
%
%\input{\bbobdatapath\algsfolder pptables_f042_20D}
%
%\input{\bbobdatapath\algsfolder pptables_f043_20D}
%
%\input{\bbobdatapath\algsfolder pptables_f044_20D}
%
%\input{\bbobdatapath\algsfolder pptables_f045_20D}
%
%\input{\bbobdatapath\algsfolder pptables_f046_20D}
%
%\input{\bbobdatapath\algsfolder pptables_f047_20D}
%
%\input{\bbobdatapath\algsfolder pptables_f048_20D}
%
%\input{\bbobdatapath\algsfolder pptables_f049_20D}
%
%\input{\bbobdatapath\algsfolder pptables_f050_20D}
%
%\input{\bbobdatapath\algsfolder pptables_f051_20D}
%
%\input{\bbobdatapath\algsfolder pptables_f052_20D}
%
%\input{\bbobdatapath\algsfolder pptables_f053_20D}
%
%\input{\bbobdatapath\algsfolder pptables_f054_20D}
%
%\input{\bbobdatapath\algsfolder pptables_f055_20D}
%
%\end{tabularx}
%
%\end{minipage}}
%
% \caption{\label{tab:aRTs20}
% \bbobpptablesmanylegend{dimension $20$}
% }
%\end{table*}
%
%
%
%%%%%%%%%%%%%%%%%%%%%%%%%%%%%%%%%%%%%%%%%%%%%%%%%%%%%%%%%%%%%%%%%%%%%%%%%%%%%%%%
%%%%%%%%%%%%%%%%%%%%%%%%%%%%%%%%%%%%%%%%%%%%%%%%%%%%%%%%%%%%%%%%%%%%%%%%%%%%%%%%
%
\bibliographystyle{ACM-Reference-Format}
\bibliography{bbob}  % bbob.bib is the name of the Bibliography in this case
%
%%%%%%%%%%%%%%%%%%%%%%%%%%%%%%%%%%%%%%%%%%%%%%%%%%%%%%%%%%%%%%%%%%%%%%%%%%%%%%%%%%%%%%%%%%%%
%
%\clearpage % otherwise the last figure might be missing
%\appendix
%
%The following pages present additional results that do not fit into the actual paper due to the page limit.
%
%
%%%%%%%%%%%%%%%%%%%%%%%%%%%%%%%%%%%%%%%%%%%%%%%%%%%%%%%%%%%%%%%%%%%%%%%%%%%%%%%%
%%%%%%%%%%%%%%%%%%%%%%%%%%%%%%%%%%%%%%%%%%%%%%%%%%%%%%%%%%%%%%%%%%%%%%%%%%%%%%%%
%
%% Scaling of aRT with dimension
%
%%%%%%%%%%%%%%%%%%%%%%%%%%%%%%%%%%%%%%%%%%%%%%%%%%%%%%%%%%%%%%%%%%%%%%%%%%%%%%%%
%\begin{figure*}
%\begin{tabular}{@{\hspace*{-0.0\textwidth}}l@{\hspace*{-0.0\textwidth}}l@{\hspace*{-0.0\textwidth}}l@{\hspace*{-0.0\textwidth}}l@{\hspace*{-0.0\textwidth}}l@{\hspace*{-0.0\textwidth}}}
%\includegraphics[width=0.2\textwidth]{ppfigs_f001}&
%\includegraphics[width=0.2\textwidth]{ppfigs_f002}&
%\includegraphics[width=0.2\textwidth]{ppfigs_f003}&
%\includegraphics[width=0.2\textwidth]{ppfigs_f004}&
%\includegraphics[width=0.2\textwidth]{ppfigs_f005}\\
%\includegraphics[width=0.2\textwidth]{ppfigs_f006}&
%\includegraphics[width=0.2\textwidth]{ppfigs_f007}&
%\includegraphics[width=0.2\textwidth]{ppfigs_f008}&
%\includegraphics[width=0.2\textwidth]{ppfigs_f009}&
%\includegraphics[width=0.2\textwidth]{ppfigs_f010}\\
%\includegraphics[width=0.2\textwidth]{ppfigs_f011}&
%\includegraphics[width=0.2\textwidth]{ppfigs_f012}&
%\includegraphics[width=0.2\textwidth]{ppfigs_f013}&
%\includegraphics[width=0.2\textwidth]{ppfigs_f014}&
%\includegraphics[width=0.2\textwidth]{ppfigs_f015}\\
%\includegraphics[width=0.2\textwidth]{ppfigs_f016}&
%\includegraphics[width=0.2\textwidth]{ppfigs_f017}&
%\includegraphics[width=0.2\textwidth]{ppfigs_f018}&
%\includegraphics[width=0.2\textwidth]{ppfigs_f019}&
%\includegraphics[width=0.2\textwidth]{ppfigs_f020}\\
%\includegraphics[width=0.2\textwidth]{ppfigs_f021}&
%\includegraphics[width=0.2\textwidth]{ppfigs_f022}&
%\includegraphics[width=0.2\textwidth]{ppfigs_f023}&
%\includegraphics[width=0.2\textwidth]{ppfigs_f024}&
%\includegraphics[width=0.2\textwidth]{ppfigs_f025}\\
%\includegraphics[width=0.2\textwidth]{ppfigs_f026}&
%\includegraphics[width=0.2\textwidth]{ppfigs_f027}&
%\includegraphics[width=0.2\textwidth]{ppfigs_f028}&
%\includegraphics[width=0.2\textwidth]{ppfigs_f029}&
%\includegraphics[width=0.2\textwidth]{ppfigs_f030}
%\end{tabular}
%\vspace{-3ex}
%\caption[Expected running time divided by dimension
%         versus dimension]{
%         \label{fig:scaling1}
%         \bbobppfigslegend{$f_1$ and $f_{30}$}
%}
%\end{figure*}
%
%\begin{figure*}
%\begin{tabular}{@{\hspace*{-0.0\textwidth}}l@{\hspace*{-0.0\textwidth}}l@{\hspace*{-0.0\textwidth}}l@{\hspace*{-0.0\textwidth}}l@{\hspace*{-0.0\textwidth}}l@{\hspace*{-0.0\textwidth}}}
%\includegraphics[width=0.2\textwidth]{ppfigs_f031}&
%\includegraphics[width=0.2\textwidth]{ppfigs_f032}&
%\includegraphics[width=0.2\textwidth]{ppfigs_f033}&
%\includegraphics[width=0.2\textwidth]{ppfigs_f034}&
%\includegraphics[width=0.2\textwidth]{ppfigs_f035}\\
%\includegraphics[width=0.2\textwidth]{ppfigs_f036}&
%\includegraphics[width=0.2\textwidth]{ppfigs_f037}&
%\includegraphics[width=0.2\textwidth]{ppfigs_f038}&
%\includegraphics[width=0.2\textwidth]{ppfigs_f039}&
%\includegraphics[width=0.2\textwidth]{ppfigs_f040}\\
%\includegraphics[width=0.2\textwidth]{ppfigs_f041}&
%\includegraphics[width=0.2\textwidth]{ppfigs_f042}&
%\includegraphics[width=0.2\textwidth]{ppfigs_f043}&
%\includegraphics[width=0.2\textwidth]{ppfigs_f044}&
%\includegraphics[width=0.2\textwidth]{ppfigs_f045}\\
%\includegraphics[width=0.2\textwidth]{ppfigs_f046}&
%\includegraphics[width=0.2\textwidth]{ppfigs_f047}&
%\includegraphics[width=0.2\textwidth]{ppfigs_f048}&
%\includegraphics[width=0.2\textwidth]{ppfigs_f049}&
%\includegraphics[width=0.2\textwidth]{ppfigs_f050}\\
%\includegraphics[width=0.2\textwidth]{ppfigs_f051}&
%\includegraphics[width=0.2\textwidth]{ppfigs_f052}&
%\includegraphics[width=0.2\textwidth]{ppfigs_f053}&
%\includegraphics[width=0.2\textwidth]{ppfigs_f054}&
%\includegraphics[width=0.2\textwidth]{ppfigs_f055}
%\end{tabular}
%\vspace{-3ex}
%\caption[Expected running time divided by dimension
%         versus dimension]{
%         \label{fig:scaling2}
%         \bbobppfigslegend{$f_{31}$ and $f_{55}$}
%}
%\end{figure*}
%
%\IfFileExists{\bbobdatapath\algsfolder ppfig2_f001.pdf}{}{\end{document}}
%
%%%%%%%%%%%%%%%%%%%%%%%%%%%%%%%%%%%%%%%%%%%%%%%%%%%%%%%%%%%%%%%%%%%%%%%%%%%%%%%%
%%%%%%%%%%%%%%%%%%%%%%%%%%%%%%%%%%%%%%%%%%%%%%%%%%%%%%%%%%%%%%%%%%%%%%%%%%%%%%%%
%% Scatter plots per function.
%%%%%%%%%%%%%%%%%%%%%%%%%%%%%%%%%%%%%%%%%%%%%%%%%%%%%%%%%%%%%%%%%%%%%%%%%%%%%%%%
%%%%%%%%%%%%%%%%%%%%%%%%%%%%%%%%%%%%%%%%%%%%%%%%%%%%%%%%%%%%%%%%%%%%%%%%%%%%%%%%
%\newcommand{\rot}[2][2.5]{
%  \hspace*{-3.5\baselineskip}%
%  \begin{rotate}{90}\hspace{#1em}#2
%  \end{rotate}}
%%%%%%%%%%%%%%%%%%%%%%%%%%%%%%%%%%%%%%%%%%%%%%%%%%%%%%%%%%%%%%%%%%%%%%%%%%%%%%%%
%%%%%%%%%%%%%%%%%%%%%%%%%%%%%%%%%%%%%%%%%%%%%%%%%%%%%%%%%%%%%%%%%%%%%%%%%%%%%%%%
%\begin{figure*}
%\centering
%\begin{tabular}{*{5}{@{}c@{}}}
%    \includegraphics[width=0.20\textwidth]{ppscatter_f001}&
%    \includegraphics[width=0.20\textwidth]{ppscatter_f002}&
%    \includegraphics[width=0.20\textwidth]{ppscatter_f003}& 
%    \includegraphics[width=0.20\textwidth]{ppscatter_f004}&
%    \includegraphics[width=0.20\textwidth]{ppscatter_f005}\\
%    \includegraphics[width=0.20\textwidth]{ppscatter_f006}&
%    \includegraphics[width=0.20\textwidth]{ppscatter_f007}&
%    \includegraphics[width=0.20\textwidth]{ppscatter_f008}&
%    \includegraphics[width=0.20\textwidth]{ppscatter_f009}&
%    \includegraphics[width=0.20\textwidth]{ppscatter_f010}\\
%    \includegraphics[width=0.20\textwidth]{ppscatter_f011}&
%    \includegraphics[width=0.20\textwidth]{ppscatter_f012}&
%    \includegraphics[width=0.20\textwidth]{ppscatter_f013}&
%    \includegraphics[width=0.20\textwidth]{ppscatter_f014}&
%    \includegraphics[width=0.20\textwidth]{ppscatter_f015}\\
%    \includegraphics[width=0.20\textwidth]{ppscatter_f016}&
%    \includegraphics[width=0.20\textwidth]{ppscatter_f017}&
%    \includegraphics[width=0.20\textwidth]{ppscatter_f018}&
%    \includegraphics[width=0.20\textwidth]{ppscatter_f019}&
%    \includegraphics[width=0.20\textwidth]{ppscatter_f020}\\
%    \includegraphics[width=0.20\textwidth]{ppscatter_f021}&
%    \includegraphics[width=0.20\textwidth]{ppscatter_f022}&
%    \includegraphics[width=0.20\textwidth]{ppscatter_f023}&
%    \includegraphics[width=0.20\textwidth]{ppscatter_f024}&
%		\includegraphics[width=0.20\textwidth]{ppscatter_f025}\\
%		\includegraphics[width=0.20\textwidth]{ppscatter_f026}&
%    \includegraphics[width=0.20\textwidth]{ppscatter_f027}&
%    \includegraphics[width=0.20\textwidth]{ppscatter_f028}&
%    \includegraphics[width=0.20\textwidth]{ppscatter_f029}&
%		\includegraphics[width=0.20\textwidth]{ppscatter_f030}
%\end{tabular}
%\caption{\label{fig:scatterplots}
%\bbobppscatterlegend{$f_1$--$f_{30}$}
%}
%\end{figure*}
%
%\begin{figure*}
%\centering
%\begin{tabular}{*{5}{@{}c@{}}}
%    \includegraphics[width=0.20\textwidth]{ppscatter_f031}&
%    \includegraphics[width=0.20\textwidth]{ppscatter_f032}&
%    \includegraphics[width=0.20\textwidth]{ppscatter_f033}& 
%    \includegraphics[width=0.20\textwidth]{ppscatter_f034}&
%    \includegraphics[width=0.20\textwidth]{ppscatter_f035}\\
%    \includegraphics[width=0.20\textwidth]{ppscatter_f036}&
%    \includegraphics[width=0.20\textwidth]{ppscatter_f037}&
%    \includegraphics[width=0.20\textwidth]{ppscatter_f038}&
%    \includegraphics[width=0.20\textwidth]{ppscatter_f039}&
%    \includegraphics[width=0.20\textwidth]{ppscatter_f040}\\
%    \includegraphics[width=0.20\textwidth]{ppscatter_f041}&
%    \includegraphics[width=0.20\textwidth]{ppscatter_f042}&
%    \includegraphics[width=0.20\textwidth]{ppscatter_f043}&
%    \includegraphics[width=0.20\textwidth]{ppscatter_f044}&
%    \includegraphics[width=0.20\textwidth]{ppscatter_f045}\\
%    \includegraphics[width=0.20\textwidth]{ppscatter_f046}&
%    \includegraphics[width=0.20\textwidth]{ppscatter_f047}&
%    \includegraphics[width=0.20\textwidth]{ppscatter_f048}&
%    \includegraphics[width=0.20\textwidth]{ppscatter_f049}&
%    \includegraphics[width=0.20\textwidth]{ppscatter_f050}\\
%    \includegraphics[width=0.20\textwidth]{ppscatter_f051}&
%    \includegraphics[width=0.20\textwidth]{ppscatter_f052}&
%    \includegraphics[width=0.20\textwidth]{ppscatter_f053}&
%    \includegraphics[width=0.20\textwidth]{ppscatter_f054}&
%		\includegraphics[width=0.20\textwidth]{ppscatter_f055}
%\end{tabular}
%\caption{\label{fig:scatterplots}
%\bbobppscatterlegend{$f_{31}$--$f_{55}$}
%}
%\end{figure*}

\end{document}



