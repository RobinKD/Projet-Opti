\documentclass{beamer}
\usetheme{Copenhagen}
\usepackage{float}
\beamertemplatenavigationsymbolsempty
\usepackage[font=footnotesize]{caption}
\captionsetup[figure]{labelformat=empty}

\usepackage[utf8]{inputenc}
\usepackage{framed}
\usepackage[super]{nth}


\title{8 novembre 2018 - \insertframenumber/\inserttotalframenumber}
%\subtitle{Using Beamer}
%
%\begin{figure}[!htb]
%    \centering
%    \begin{minipage}{.5\textwidth}
%        \centering
%        \includegraphics[scale=0.05]{logo-ensta.png}
%    \end{minipage}%
%    \begin{minipage}{0.5\textwidth}
%        \centering
%        \includegraphics[scale=0.12]{logo-ens.jpg}
%    \end{minipage}
%\end{figure}
%
%{\vfill Martin BAUW\\\textit{under the supervision of} Thibault TORALBA}
%\end{center}
%
\author{TC2 Project}
%\institute{University of ShareLaTeX}
%\date{\today}

%\date{mercredi 6 septembre 2017}

\begin{document}

\begin{frame}
\begin{center}
{ M2 AIC\\\footnotesize TC2: Introduction to Optimization}
\vfill
{\large
\begin{framed}
Black-Box Optimization Benchmarking\\ with the COCO platform\\ - \\Multiobjective Optimizer adaptive IBEA ($\epsilon$-indicator) 
\end{framed}
}
\vfill
\begin{figure}
\centering
\includegraphics[scale=0.25]{logo_upsaclay.jpg}
\end{figure}

{\footnotesize \vfill \textit{Group 1:} Martin BAUW, Robin DURAZ, Jiaxin GAO,\\ Hao LIU, Luca VEYRON-FORRER}
\end{center}
\end{frame}

\begin{frame}
\tableofcontents
\end{frame}

\section{Introduction}
\begin{frame}

\end{frame}


\section{The algorithm}
\subsection{Overview of IBEA}
\begin{frame}
\begin{itemize}
\item steps description
\end{itemize}
\end{frame}


\begin{frame}
\frametitle{Binary quality indicators}

\end{frame}

\subsection{Selection and variation}
\begin{frame}
\frametitle{Mating selection and mutation}

\end{frame}

\begin{frame}
\frametitle{Recombination}

\end{frame}

\section{Our implementation}
\subsection{Code structure}
\begin{frame}

\end{frame}

\subsection{Improvements regarding the execution speed}
\begin{frame}

\end{frame}

\section{CPU timing and results}
\subsection{CPU timing and results}
\begin{frame}

\end{frame}

\subsection{Comparision with NSGA 2 and Random Search}
\begin{frame}
\frametitle{NSGA 2}

\end{frame}

\begin{frame}
\frametitle{Random Search}

\end{frame}

\section{Conclusion}
\begin{frame}
conclusion sur algo, implémentation, idées d'ouverture ?
\end{frame}

\section{Bibliography}
\begin{frame}
\frametitle{Non-exhaustive bibliography}

\end{frame}

\end{document}