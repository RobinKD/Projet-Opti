\documentclass{beamer}
\usetheme{Copenhagen}
\usepackage{float}
\beamertemplatenavigationsymbolsempty
\usepackage[font=footnotesize]{caption}
\captionsetup[figure]{labelformat=empty}

\usepackage[utf8]{inputenc}
\usepackage{framed}
\usepackage[super]{nth}

\usepackage{tabularx}
\usepackage{xspace}
\usepackage{multirow}% http://ctan.org/pkg/multirow
\usepackage{diagbox}
\usepackage{changepage}

\title{8 novembre 2018 - \insertframenumber/\inserttotalframenumber}
%\subtitle{Using Beamer}
%
%\begin{figure}[!htb]
%    \centering
%    \begin{minipage}{.5\textwidth}
%        \centering
%        \includegraphics[scale=0.05]{logo-ensta.png}
%    \end{minipage}%
%    \begin{minipage}{0.5\textwidth}
%        \centering
%        \includegraphics[scale=0.12]{logo-ens.jpg}
%    \end{minipage}
%\end{figure}
%
%{\vfill Martin BAUW\\\textit{under the supervision of} Thibault TORALBA}
%\end{center}
%
\author{TC2 Project}
%\institute{University of ShareLaTeX}
%\date{\today}

%\date{mercredi 6 septembre 2017}

\begin{document}

\begin{frame}
\begin{center}
{ M2 AIC\\\footnotesize TC2: Introduction to Optimization}
\vfill
{\large
\begin{framed}
Black-Box Optimization Benchmarking\\ with the COCO platform\\ - \\Multiobjective Optimizer adaptive IBEA ($\epsilon$-indicator) 
\end{framed}
}
\vfill
% \begin{figure}
% \centering
% \includegraphics[scale=0.25]{logo_upsaclay.jpg}
% \end{figure}

{\footnotesize \vfill \textit{Group 1:} Martin BAUW, Robin DURAZ, Jiaxin GAO,\\ Hao LIU, Luca VEYRON-FORRER}
\end{center}
\end{frame}

\begin{frame}
\tableofcontents
\end{frame}

\section{Introduction}
\begin{frame}
IBEA: Indicator-Based Evolutionary Algorithm
\begin{itemize}
\item optimization: find decision space vectors leading to objective space minima
\item multiobjective: the objective space is multidimensional
\item evolutionary: decision space candidates follows an natural selection-like evolution
\item indicator-based: binary quality indicators to compare two Pareto set approximations
\end{itemize}
\end{frame}


\section{The algorithm}
\subsection{Overview of IBEA}
\begin{frame}
\begin{columns}[T]
\begin{column}{.48\textwidth}
Successive steps of IBEA:
\begin{enumerate}
\item Initialization
\item Fitness assignment % scale each objective to [0,1], calculate Is then c then fitness values
\item Environmental selection % update fitness values thanks to F()
\item Termination
\item Mating selection
\item Variation
\end{enumerate}
\end{column}
\begin{column}{.48\textwidth}
\includegraphics[scale=0.16]{stochasticSearchAlgorithm.png} \footnotemark
\end{column}
\end{columns}
\footnotetext[1]{Illustration from:\\ \textit{A Tutorial on Evolutionary Multiobjective
Optimization} - E. Zitzler, \\M. Laumanns, and S. Bleuler}
\end{frame}


\begin{frame}
\begin{adjustwidth}{-2.2em}{-2.2em}
\begin{itemize}
\item \underline{Binary quality indicators:}
\begin{equation}
I_{\epsilon^+ (A,B)} = min_{\epsilon} \{\forall x^2 \in B\ \exists x^1 \in A : f_i(x^1) - \epsilon \leq f_i (x^2)\ for\ i \in \{1,...,n\}\}
\end{equation}
\item \underline{Fitness values:}
\begin{equation}
F(x^1) = \sum_{x^2 \in P \backslash \{x^1\}} -e^{-\dfrac{I(\{x^1\},\{x^2\})}{ck}}
\end{equation}
\end{itemize}
\end{adjustwidth}
\begin{center}
\includegraphics[scale=0.2]{binaryIndicators.png} \footnotemark
\end{center}
\footnotetext[2]{Illustration from:\\ \textit{Indicator-Based Selection in Multiobjective Search} - E. Zitzler and S. Künzli}
\end{frame}

\subsection{Selection and variation}

\begin{frame}
\frametitle{Mating selection}
Binary tournament selection
\begin{itemize}
    \item Two individuals randomly chosen from the population
    \item Best individual kept in mating pool
    \item Repeated until mating pool filled
\end{itemize}
\footnotetext[3]{\textit{A Tutorial on Evolutionary Multiobjective
Optimization} - E. Zitzler, \\M. Laumanns, and S. Bleuler}
\end{frame}

\begin{frame}
\frametitle{Recombination}
For recombination, Simulated Binary Crossover (SBX) operator was chosen.  
A random number $u$ created within $[0,1]$, as follows:
\begin{itemize}
\item if $u$ $\leq 0.5$:
\begin{equation}
\beta_q = (2u)^{\frac{1}{\eta_c +1}}
\end{equation}
\item else:
\begin{equation}
\beta_q = (\frac{1}{2(1-u)})^{\frac{1}{\eta_c +1}}
\end{equation}
\end{itemize}
\footnotetext[1]{\textit{K. Deb and R. B. Agrawal. Simulated binary
crossover for continuous search space. Complex
Systems, 9(2):115–148, 1995.} }
\end{frame}

\begin{frame}
\frametitle{Recombination}
Thus, we can compute the children's coordinates:

\begin{itemize}
\item first child:
\begin{equation}
child0[j] = 0.5((1+\beta_q)parent0[j]+(1-\beta_q)parent1[j])
\end{equation}
\item second child:
\begin{equation}
child1[j] = 0.5((1-\beta_q)parent0[j]+(1+\beta_q)parent1[j])
\end{equation}
\footnotetext[3]{\textit{K. Deb and R. B. Agrawal. Simulated binary
crossover for continuous search space. Complex
Systems, 9(2):115–148, 1995.} }

\end{itemize}

\end{frame}

\begin{frame}
\frametitle{Mutation}
Polynomial mutation operator:
this mutation operator modifies individuals by changing small parts in the associated vectors according to a given mutation rate.

\begin{itemize}
\item if $u$ $\leq 0.5$:
\begin{equation}
\sigma_L = (2u)^{\frac{1}{\eta_m +1}}-1 
\end{equation}
\begin{equation}
p_{mut}[j] = ind[j] + \sigma_L(ind[j]-Lo)
\end{equation}
\item else:
\begin{equation}
\sigma_R = (2(1-u))^{\frac{1}{\eta_m +1}} 
\end{equation}
\begin{equation}
p_{mut}[j] = ind[j] + \sigma_R(Up-ind[j])
\end{equation}
\footnotetext[4]{\textit{K. Deb and S. Agrawal. A niched-penalty approach
for constraint handling in genetic algorithms. In Parallel Problem Solving
from Nature (PPSN-VI), pages 365–374, 2000.} }
\end{itemize}

\end{frame}

\section{Our implementation}
\subsection{Code structure}
\begin{frame}
\begin{itemize}
\item Code built for the most general case
\item The IBEA code is in the class IBEA, where each method implements one step of the algorithm
\item No difficulty to get to the best asymptotic complexity
\end{itemize}
\end{frame}

\subsection{Improvements regarding the execution speed}
\begin{frame}
\begin{itemize}
\item Good data structures choices
\item The Indicator function was the key
\item Execution time improvement : 59.6s to 12.7s per execution (divided by 4.6)

\end{itemize}
\end{frame}

\section{CPU timing and results}
\subsection{CPU timing and results}
\begin{frame}
  \frametitle{Computer specifications and batch options}
  \begin{itemize}
  \item Intel(R) Core(TM) i7-7500U CPU @ 2.70GHz
  \item Quad core CPU with 16GB RAM
  \end{itemize}
  \vspace{1em}
  Everything ran with a budget of 100\\
  \begin{itemize}
  \item Three batchs for dimensions 2, 3, 5, 10, 20
  \item First batch running alone, and two others together
  \item One batch for dimensions 40
  \end{itemize}
\end{frame}

\begin{frame}
  \frametitle{Options chosen to run the algorithm}
  \begin{itemize}
  \item Population size : 100
  \item Maximum number of generation : 100
  \item Scaling factor : 0.05
  \item Mutation rate : 0.01
  \item Recombination and mutation $\eta_m$ \& $\eta_c$ : 1
  \item Population initialization in range (-5, 5)
  \end{itemize}
\end{frame}

\begin{frame}
  %\frametitle{Time by function evaluation}
  \begin{table}[!htbp]
    \begin{tabular}{|p{3.8cm}|p{1.5cm}|p{1.5cm}|p{1.5cm}|}
      \hline
      \diagbox{Batch}{Dimension} & 2 & 3 & 5\\
      \hline
      Batch 1 on 3 & 6.0e-04 & 6.3e-04 & 8.1e-04\\
      \hline
      \multirow{2}{3.8cm}{Batch 2 and 3 on 3 run simultaneously} & 8.6e-04 & 8.6e-04 & 9.1e-04\\
      \cline{2-4}
                                 & 8.3e-04 & 8.4e-04 & 8.9e-04\\
      \hline
    \end{tabular}
    \begin{tabular}{|p{3.8cm}|p{2.47cm}|p{2.47cm}|}
      \hline
      \diagbox{Batch}{Dimension} & 10 & 20\\
      \hline
      Batch 1 on 3 & 8.3e-04 & 1.1e-03\\
      \hline
      \multirow{2}{3.8cm}{Batch 2 and 3 on 3 run simultaneously} & 1.1e-03 & 1.3e-03\\
      \cline{2-3}
                                 & 1.0e-03 & 1.3e-03\\
      \hline
    \end{tabular}
    \begin{tabular}{|p{3.8cm}|p{5.37cm}|}
      \hline
      \diagbox{Batch}{Dimension} & 40\\
      \hline
      Whole test suite & 4.2e-03\\
      \hline
    \end{tabular}
  \end{table}
\end{frame}


\begin{frame}
  \frametitle{Results}
  \begin{figure}
    \centering
    \includegraphics[scale=0.5]{pprldmany}
  \end{figure}
\end{frame}

\begin{frame}
  \frametitle{Results analysis}
  \begin{itemize}
  \item Comparatively better in higher dimensions
  \item Results globally good for a MOEA
  \end{itemize}
  \begin{itemize}
  \item More budget could have given better results
  \item A better initialization of population could lead to a sharper increase
    at the beginning
  \end{itemize}
\end{frame}

\subsection{Comparison with Random Search and NSGA-II}

\begin{frame}
\frametitle{Random Search}
\begin{itemize}
\item Its ECDF looks like linear functions
\item It doesn't work well when the dimension is too high
\item Globally worse than our algorithm
\end{itemize}
\centering \includegraphics[scale=0.35]{Random.png}
\end{frame}

\begin{frame}
\frametitle{NSGA-II}
Algorithm faster, although budget is much higher.\\
It has slightly better results.\\
\centering \includegraphics[scale=0.35]{NSGA.png}

\end{frame}

\section{Conclusion}
\begin{frame}
\begin{itemize}
    \item Overall satisfaction with our results.
    \item Parameter tuning could be further studied.
    \item Modification of small modules of MOEA.
\end{itemize}
\end{frame}

\section{Bibliography}
\begin{frame}
\frametitle{Non-exhaustive bibliography}
\begin{itemize}
\item \textit{Indicator-Based Selection in Multiobjective Search} - Zitzler, E. and Künzli, S.
\item \textit{A Tutorial on Evolutionary Multiobjective Optimization} - Zitzler, E. and Laumanns, M. and Bleuler, S.
\item \textit{Biobjective Performance Assessment with the {COCO} Platform} - Brockhoff, D. and Tu{\v s}ar, T. and Tu{\v s}ar, D. and Wagner, T. and Hansen, N. and Auger, A.
\end{itemize}
\end{frame}

\end{document}
