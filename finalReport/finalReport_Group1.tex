 \documentclass[12pt]{article}

\usepackage{framed}
\usepackage[super]{nth}
\usepackage{float}
\usepackage{caption}
\usepackage{commath}


\usepackage{fancyhdr} % Required for custom headers
\usepackage{lastpage} % Required to determine the last page for the footer
\usepackage{extramarks} % Required for headers and footers
\usepackage[usenames,dvipsnames]{color} % Required for custom colors
\usepackage{graphicx} % Required to insert images
\usepackage{lipsum} % Used for inserting dummy 'Lorem ipsum' text into the template
\usepackage{courier} % Required for the courier font
\usepackage{array}
\usepackage{pbox}
\usepackage{xcolor}

\usepackage[utf8]{inputenc}

\usepackage{amsmath}

% Margins
\topmargin=-0.45in
\evensidemargin=0in
\oddsidemargin=0in
\textwidth=6.5in
\textheight=9.0in
\headsep=0.25in

\linespread{1.1} % Line spacing
\date{}

\begin{document}

\title{Adaptive-IBEA with epsilon indicator implementation and benchmarking via the COCO platform \\ \-- \\ M2 AIC Introduction to Optimization project}
\author{Martin BAUW, Robin D, Jiaxin G,\\ Hao LIU, and Luca VF}
\maketitle

\tableofcontents

\newpage

\section{Abstract}

This report describes the implementation of an adaptive-IBEA with epsilon indicator (iteration-based evolutionary algorithm) in Python and its benchmarking with the COCO platform. The implemented algorithm is part of the multiobjective evolutionary algorithms (MOEAs). It aims at taking into account user preferences without requiring diversity preservation mechanism.

\section{Introduction}

\section{Adaptive-IBEA with epsilon indicator}

\section{Python implementation}

\section{Timing experiment}

\section{Results}

\section{Conclusion}

\end{document}
